\documentclass[12]{resume} % Use the custom resume.cls style

\usepackage[left=0.75in,top=0.6in,right=0.75in,bottom=0.6in]{geometry} % Document margins
\usepackage{graphicx}

\begin{document}
\begin{rSection}
{\Large \textbf{Gelfond's Constant}}\vspace{1em}
\end{rSection}


Gelfonds Constant is a Transcendental Number,  a Transcendental Number is a number which is not an Algebraic number. This constant is named after Aleksandr Gelfond and it is represented as
$e^\pi$  that is e raised to the power Pie.


The value of Gelfonds constant is:
$$e^\pi  =23.14069263277926900572908636794854738………$$

     Where     $e^\pi= {e^{\pi} i}-i= (-1)^-i$\\
Here i is imaginary, Since -i is not algebraic, we can say $e^\pi$ is Transcendental 


\begin{rSection}{USAGE:}
This constant is used in \\
 * Gelfond–Schneider theorem\\
* Hilbert's seventh problem\\
* Ramanjuna’s Constant\\

\end{rSection}

\begin{rSection}{CONSTRUCTION:}
Assume   $\displaystyle{k_0=  \frac{2}{3}}$  and\\
$\displaystyle{k_n+1 = \frac{1-\sqrt{1-{k_n}^2}}{1-\sqrt{1+{k_n}^2}}}s$
  

When $n>0$ the sequence expands and becomes\\

{\Large$$e^\pi = 23 + \frac{1}{7 + \frac{1}{9 + \frac{1}{3 + \frac{1}{1 + \frac{1}{1 + \frac{1}{591 + \frac{1}{...}}}}}}}$$}

\end{rSection}

\end{document}
