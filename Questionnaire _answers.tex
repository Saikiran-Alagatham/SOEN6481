\documentclass[12pt]{article}
\usepackage[utf8]{inputenc}
\usepackage{natbib}

\title{\textbf{\huge Gelfond's Constant\vspace{-4ex}}% to see the effect
}
\usepackage[left=0.75in,top=0.6in,right=0.75in,bottom=0.6in]{geometry} 
\usepackage{graphicx}
\setlength{\parskip}{1em}

\date{}

\begin{document}

    \maketitle
\section*{Introduction\vspace{-2ex}}
Gelfond's Constant is a Transcendental Number,  a Transcendental Number is a number which is not an Algebraic number. This constant is named after Aleksandr Gelfond and it is represented as
$e^\pi$  that is e raised to the power Pie.\\

The value of Gelfond's constant is:
$$e^\pi  =23.14069263277926900572908636794854738………$$

Where     $e^\pi= {e^{\pi} i}-i= (-1)^-i$ \par

Here i is imaginary, Since -i is not algebraic, we can say $e^\pi$ is Transcendental.

\section*{Usage\vspace{-2ex}}
This constant is used in \\
    $* $ Gelfond–Schneider theorem\\
    $* $ Hilbert's seventh problem\\
    $* $ Ramanjuna’s Constant\\

\section*{Construction\vspace{-2ex}}
Assume   $\displaystyle{k_0=  \frac{2}{3}}$  and\\
$\displaystyle{k_n+1 = \frac{1-\sqrt{1-{k_n}^2}}{1-\sqrt{1+{k_n}^2}}}s$\\

When $n>0$ the sequence expands and becomes\\
{\Large$$e^\pi = 23 + \frac{1}{7 + \frac{1}{9 + \frac{1}{3 + \frac{1}{1 + \frac{1}{1 + \frac{1}{591 + \frac{1}{...}}}}}}}$$}

\section*{\centering Interview}

\textbf{1.What best defines you?}\\
 I’m Dr D Srinivas Reddy, a Professor at Jawaharlal Nehru Technical University Hyderabad. I have 15 years of experience in this field , worked in various collages and universities across India. \textbf{Fluid Mechanics} and \textbf{Mathematical Model} , these are my areas of specialization.
 
\noindent \textbf{2.Numbers and constants( e.g pie) , How important are they in your field?}\\
According to me ‘e’ and ‘pie’ are most important constants in mathematical field.
‘e’ is used mainly in Accounts and Banking sector , Where as ‘Pie’ in Geometry. One would assume they would get more interest for a sum per year, compared to six months or 3 months or each month. ‘e’ helped to clear that assumptions.

\noindent \textbf{3.What do u know about Gelfond's constant?}\\
Not Much, apart from that it used in gelfond-schneider theorem and hilbert's seventh problem.

\noindent \textbf{4.What are the uses of Gelfond's constant and what will the future?}\\
The same hilbert's seventh problem and gelfond-schneider theorem. Regarding future of the constant, I think it pretty much remain the same.

\noindent \textbf{5.What are the requirements to build a calculator to perform operations on  constants?}\\
The main factor would be the precision , especially when it comes to multiplying constants and presenting the result with appropriate decimal point .

\noindent \textbf{6.Which the most widely used constant and what is its  significance?}\\
I would pick Pie, because of its huge impact in geometrical field. The other would be golden ratio(1.6168) , even this belong to the same family of Pie.

\noindent \textbf{7.What device do you generally use for the complex calculations in your field?}\\
A regular calculator and a Scientific Calculator. If it's really different I use a site called  ‘vCalc’, it offers a huge range of scientific formula related calculators.

\noindent \textbf{8.What kind of user interface would you like for the device?}\\
A simple regular interface, with standard layout of buttons, because if its personalized to yourself then we can’t build a standard calculator which can be used all.

\noindent \textbf{9.What are the features u wish u had in a modern calculator?}\\
I want to add personalized buttons which I use mostly during calculations, that being said  it will not become a standard calculator. So A button to store a value that I want to store will be fine.








\end{document}
